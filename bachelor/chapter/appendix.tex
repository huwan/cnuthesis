\chapter{附$\quad$录}
\label{chap:appendix}
%%每个section是一个附录。
%%附录一
\section{首都师范大学本科生毕业论文(设计)写作规范}
\begin{itemize}[font=\cusong]
\item[一、] {\cusong 毕业论文(设计)写作的基本要求}
\begin{enumerate}
\item 毕业论文(设计)应采用国家正式公布实施的简化汉字和法定的计量单位。
\item 毕业论文(设计)中采用的术语、符号、代号必须统一,并符合规范化的要求。使用新的专业术语、缩略语、习惯用语等,应加以注释。国外新的专业术语、缩略语,必须在译文后用小括号注明原文。
\item 毕业论文(设计)中的图和表应有对应的图题、表题及编号。
	\begin{enumerate}[label=(\arabic* )]
		\item 
图:由“图”和从1开始的阿拉伯数字组成,例如“图1”、“图2”等。图的编号应一直连续到附录之前,与章、节的编号无关。只有一幅图时,仍应标为“图1”。图应有图题,表明本图的主题,空一格置于图的编号之后,图的编号和图题应置于图下方居中的位置,字体采用宋体五号;
		\item 
表:由“表”和从1开始的阿拉伯数字组成,如“表1”、“表2”等。表的编号应一直连续到附录之前,与章、节的编号无关。只有一个表时,仍应标为“表1”。表应有表题,表明本表的主题,空一格置于表的编号之后,表的编号和表题应置于表上方的居中位置,字体采用宋体五号;
		\item 
公式序号一律采用阿拉伯数字分章依序编排;如:“式(2-13)”、“式(4-5)”,其标注应于该公式所在行的最右侧。公式书写方式应在文中相应位置另起一行居中横排,对于较长的公式只可在符号处(+、-、*、/、≤≥等)转行。
	\end{enumerate}
\item 毕业论文(设计)的正文中不应加入程序的源代码,除非该源代码为完成毕业论文、保证其内容完整性所必需。
\item  毕业论文(设计)的文档格式
	\begin{enumerate}[label=(\arabic* )]%[(1)]
\CTEXindent
		\item 首页:
		\begin{enumerate}[label=\Roman* .]%[I]
\item 论文编码:使用小四号楷体字,置顶,居右放置。
\item 文头:“首都师范大学本科生毕业论文”字样使用一号宋体字,加粗,在论文编码下两行,居中放置。
\item 论文题名:居中,隔一行,排印在论文文头下,使用小一号宋体字,加粗。
\item 论文副题名:居中排印在论文题名下,使用小二号宋体字,加粗,副题名前加特殊符号中的“长划线”。
\item 院(系)、专业、年级、学号、指导教师、论文作者、完成日期:隔五行,依次排印在论文副题名下(如无副题名须隔六行),各占一行,使用三号宋体字,加粗,距左端空7格;项目名称需要两端对齐,内容下需要加下划线,并将内容置于下划线中部。
		\end{enumerate}

		\item 中文摘要:项目名称置顶,居中放置,小二号黑体字加粗;摘要内容:小四号宋体字,起行空两格。英文摘要:项目名称置顶,居中放置,小二号Times New Roman字体加粗,摘要内容:小四号,Times New Roman字体,起行空两格;
		\item 关键词:项目名称使用四号黑体字,段首空两格;关键词内容:小四号宋体,词间用中文状态下“;”分隔。英文关键词:项目名称使用四号Times New Roman字体,段首空两格;关键词内容:小四号Times New Roman字体,词间用英文状态下“;”分隔。
		\item 目录:二号黑体,加粗;
		\begin{verbatim}
      1(章的标题) XXXX ……………………………1(三号宋体字、加粗)
      1.1 (节的标题) XXXX ……………………………2 (小三号宋体字)
      1.1.1 (条的标题) XXXX…………………………3 (四号宋体字)
      1.1.1.1 (款的标题) XXXX ………………………4(小四号宋体字)
		\end{verbatim}
		\item 正文:正文按照自然段依次排列,每段起行空两格,回行顶格。一般使用小四号宋体字,英文内容使用小四号Times New Roman字体;章、条、款、项的标题字体要求与目录相同且加粗。
		\item 参考文献: \par
项目名称使用小二号黑体字加粗,居中放置,内容按论文中参考文献出现的次序,用加中括号的数字连续编号,依次书写作者、文献名、杂志或书名、卷号或期刊号、出版时间。中外人名一律采用姓在前、名在后的著录法。五号宋体书写。正文中参照参考文献的部分,需要在文档中用右上角[1][2][3] $\cdots\cdots$的方式标明,与参考文献中的编号一一对应。
		\item 脚注:\par
内容格式与参考文献要求一致,宋体小五号书写;编号方式为\textcircled{1}\textcircled{2}\textcircled{3}$\cdots\cdots$,每页重新编号。
	\end{enumerate}
\item 学生应签署《首都师范大学本科生毕业论文(设计)原创性承诺书》(参
见《首都师范大学本科生毕业论文(设计)模板》)。保证独立完成毕业论文(设
计),不存在抄袭和剽窃。毕业论文(设计)的知识产权归学校所有。
\item 毕业论文(设计)打印的用纸要求\\
本科学生毕业论文(设计)采用标准A4型(297mm×210mm)打印纸或复印纸印制。
\item 毕业论文(设计)的排版要求
	\begin{enumerate}[label=(\arabic* )]%[(1)]
	\item 页面设置\par
\CTEXindent 
本科学生毕业论文(设计)要求纵向打印,页边距的要求为:\par

上(T):2.5 cm\par

下(B):2.5 cm\par

左(L):2.5cm\par

右(R):2 cm\par

装订线(T):0.5 cm\par

装订线位置(T):左\par

其余设置采取系统默认设置。
\item 段落设置\par
在“格式”选项中的“段落”设置窗口中,取消“如果定义了文档网格,则与网格对齐(w)”选项,采用多倍行距,行距设置值为1.25。其余设置采取系统默认设置。
\item 页眉、页脚设置\par
本科学生毕业论文(设计)的页眉使用学校标志:高度为0.98 cm,宽度为4.13 cm,居中放置。\par
本科学生毕业论文(设计)的页脚需要设置页码,设置页码时操作为:选择插入$\rightarrow$页码,“位置”选择“页面底端(页脚)”,“对齐方式”选择“外侧”,取消“首页显示页码”选项,“格式”选择“-1-,-2-,-3-……”。
	\end{enumerate}
\end{enumerate}

\item[二、]{\cusong 各部分规范的具体要求}\par
毕业论文(设计)应包括论文封面、论文题目、中英文摘要、目录、引言、论文正文、结论、参考文献等主要组成部分,具体要求如下:
\begin{enumerate}[label=\bfseries\arabic*.]
\item {\cusong 论文封面}\par
 \CTEXindent 一律采用教务处印制的统一格式的封面。
\item {\cusong 题目}\par
题目是反映论文内容的最恰当、最简明的词语组合。题目语意未尽可用副标题补充说明论文中的特定内容。要求如下:
\begin{enumerate}[label=(\arabic* )]%[(1)]
\item 题目准确得体并能准确表达论文的中心内容,恰当反映研究的范围和深度,不能使用笼统的、泛指性很强的词语和华丽不实的词藻。
\item 题目应简明,使读者印象鲜明,便于记忆和引用。题目一般不宜超过20字。
\item 题目所用词语必须有助于选定关键词和编制题录、索引等二次文献,以便为检索提供特定的实用信息。
\item 题目应避免使用非共知共用的缩略词、字符、代号等。
\end{enumerate}
\item {\cusong 摘要}\par
摘要是对论文内容不加注释和评论的简明归纳,应包括研究工作的目的、方法和结论,重点是结果和结论。用语要规范,一般不用公式和非规范符号术语,不出现图、表、化学结构式等。采用第三人称撰写,一般在300字左右。\par
论文应附有英文题目和英文摘要以便于进行国际交流。 英文题目和英文摘要应明确、简练,其内容包括研究目的、方法、主要结果和结论。一般不宜超过250个实词。
\item {\cusong 关键词}\par
关键词是为了满足文献标引或检索工作的需要而从论文中选取出的用以表示全文主题内容信息的词或词组。关键词包括主题和自由词:主题词是专门为文献的标引或检索而从自然语言的主要词汇中挑选出来并加以规范化了的词或词组;自由词则是未规范化的即还未收入主题词表中的词或词组。\par
每篇论文中应列出3~8个关键词,它们应能反映论文的主题内容。其中主题词应尽可能多一些,关键词作为论文的一个组成部分,列于摘要段之后。还应列出与中文对应的英文关键词(Key words)。
\item {\cusong 引言(或前言)}\par
引言又叫前言,其目的是向读者交代本研究的来龙去脉,作用在于使读者对论文先有一个总体的了解。引言要写得自然,概括,简洁,确切。内容主要包括:研究的目的、范围和背景;理论依据、实验基础和研究方法;预期的结果及其地位、作用和意义等。
\item {\cusong 正文}\par
      正文是论文的核心部分,占主要篇幅,论文的论点、论据和论证都在这里阐述。由于论文作者的研究工作涉及的学科、研究对象和研究方法和结果表达方式等差异很大,所以对正文的撰写内容不作统一规定。但总的思路和结构安排应当符合“提出论点,通过论据或数据对论点加以论证”这一共同的要求。正文应达到观点正确、结构完整、合乎逻辑、符合学术规范,无重大疏漏或明显的片面性。其他具体要求有:
\begin{enumerate}[label=(\arabic* )]%[(1)]
\item 主题的要求\par
\begin{itemize}[fullwidth,itemindent=2em]
\item[A.]	主题有新意,有科学研究或实际应用价值;
\item[B.]	主题集中,一篇论文只有一个中心,要使主题集中,凡与本文主题无关或关系不大的内容不应涉及,不过多阐述,否则会使问题繁杂,脉络不清,主题淡化;
\item[C.]	主题鲜明,论文的中心思想地位突出,除了在论文的题目、摘要、前言、结论部分明确地点出主题外,在正文部分更要注意突出主题。
\end{itemize}
\item 结构的要求\par
\begin{itemize}[fullwidth,itemindent=2em]
\item[A.]	不同内容的正文,应灵活处理,采用合适的结构顺序和结构层次,组织好段落,安排好材料。 章、节、小节等分别以“1”、“1.1”、“1.1.1”、“1.1.2”、等数字以树层次格式依次标出。
\item[B.]	正文写作时要注意抓住基本观点。数据的采集、记录、整理、表达等均不应出现技术性的错误;分析论证和讨论问题时,避免含混不清,模棱两可,词不达意 ;不弄虚作假。
\end{itemize}
\end{enumerate}
\item {\cusong 结论和建议}\par
结论即结束语、结语,是在理论分析和实验验证的基础上,通过严密的逻辑推理得出的有创造性、指导性、经验性的结果描述。反映了研究成果的价值,其作用是便于读者阅读和二次文献作者提供依据。主要包含本研究结果说明了什么问题,得出了什么规律性的东西,或解决了什么实际问题;本研究的不足之处、尚待解决的问题或提出研究设想和改进建议。
\item {\cusong 参考文献}\par
 应是论文作者亲自考察过的对毕业论文(设计)有参考价值的文献,除个别专业的外,均应有外文参考文献。参考文献应具有权威性,要注意引用最新的文献。\par
按照参考文献在文中出现的顺序采用阿拉伯数字连续编号,参考文献著录格式如下:
\begin{enumerate}[label=(\arabic* )]%[(1)]
\item 著作:[序号]作者1,作者2.译者.书名[文献类型标志](英文用[M]).版本.出版地:出版社,出版时间,引用部分起止页.
\item 期刊:[序号]作者1,作者2.译者.文章题目[文献类型标志](英文用[J]).期刊名,年份,卷号(期数):引用部分起止页.
\item 会议论文集:[序号]作者.译者.文章名.文集名.会址.开会年.出版地:出版社,出版时间,引用部分起止页.
\item 学位论文:[序号]作者.题名.[文献类型标志](英文用[C]).保存地点:保存单位,年份.
\item 专利:[序号]专利申请者.题名:国别,专利号[文献类型标志](英文用[P]).公告日期或公开日期,获取和访问路径.
\item 电子文献:主要责任者.题名:其他题名信息[文献类型标志/文献载体标志口] (英文用[EB/OL]) .出版地:出版者,出版年(更新或修改日期)[引用日期].获取和访问路径.
\end{enumerate} 
注:文献中的作者数量低于三位时全部列出;超过三位时只列前三位,其后加“等”字即可;作者姓名之间用逗号分开;中外人名一律采用姓在前,名在后的著录法。   

\item {\cusong  附录}\par
     附录是论文主体的补充项目,为了体现整篇论文的完整性,写入正文又可能有损于论文的条理性、逻辑性和精炼性,这些材料可以写入附录段,但对于每一篇论文并不是必须的。主要包括以下几类:
\begin{enumerate}[label=(\arabic* )]
\item 比正文更为详尽的理论根据、研究方法和技术要点,建议可以阅读的参考文献的题录,对了解正文内容有用的补充信息等;
\item 由于篇幅过长或取材于复制品而不宜写入正文的材料;
\item 一般读者并非必要阅读,但对本专业同行很有参考价值的资料;
\item 某些重要的原始数据、数学推导、程序源代码、框图、结构图、统计表、计算机打印输出件等。
\end{enumerate}\par
附录段置于参考文献表之后,附录中的插图、表格、公式、参考文献等的序号与正文分开,另行编制,如编为“图1”、“图2”;“表1”、“表2”;“式1”、“式2”;“文献1”、“文献2”等。
\item {\cusong 致谢}\par
      有些毕业论文(设计)不是一个人单独完成的,为此在必要时应增加本部分,以对论文工作直接提供过资金、设备、人力,以及文献资料等支持和帮助的团体和个人表示感谢。
\end{enumerate}
\end{itemize}

%%附录二
\newpage
\section{图表、公式等使用示例}
《首都师范大学本科生毕业论文(设计)写作规范》对附录中图表的要求:
附录中的插图、表格、公式、参考文献等的序号与正文分开,另行编制,如编为“图1”、“图2”;“表1”、“表2”;“式1”、“式2”;“文献1”、“文献2”等。\par

\begin{figure}[!hbt]
\centering
\subfigure[校徽1]{
\includegraphics[width=0.2\textwidth]{cnulogo.eps}\label{f:1}}
\subfigure[校徽2]{
\includegraphics[width=0.2\textwidth]{cnulogo.eps}\label{f:2}}
\caption{\label{f:s}首都师范大学校徽并列子图示例}
\end{figure}

浮动体的并排放置一般有两种情况:1)二者没有关系,为两个独立的浮动体;2)二者隶属
于同一个浮动体。对表格来说并排表格可以如表~\ref{tab:subtable} 使用子表格来做。

\begin{table}[htbp]
\centering
\caption{并排子表格}
\label{tab:subtable}
\subfigure[第一个子表格]{
\begin{tabular}{p{2cm}p{2cm}}
\toprule[1.5pt]
111 & 222 \\\midrule[1pt]
222 & 333 \\\bottomrule[1.5pt]
\end{tabular}}\hskip2cm
\subfigure[第二个子表格]{
\begin{tabular}{p{2cm}p{2cm}}
\toprule[1.5pt]
111 & 222 \\\midrule[1pt]
222 & 333 \\\bottomrule[1.5pt]
\end{tabular}}
\end{table}

如果您要排版的表格长度超过一页,那么推荐使用 \textsf{longtable} 或者 \textsf{supertabular} 
宏包,表~\ref{tab:performance} 就是 \textsf{longtable} 的简单示例。
\begin{longtable}[c]{c*{6}{r}}
\caption{实验数据}\label{tab:performance}\\
\toprule[1.5pt]
 测试程序 & \multicolumn{1}{c}{正常运行} & \multicolumn{1}{c}{同步}
& \multicolumn{1}{c}{检查点}   & \multicolumn{1}{c}{卷回恢复}
& \multicolumn{1}{c}{进程迁移} & \multicolumn{1}{c}{检查点} 	\\
& \multicolumn{1}{c}{时间 (s)} & \multicolumn{1}{c}{时间 (s)}
& \multicolumn{1}{c}{时间 (s)} & \multicolumn{1}{c}{时间 (s)}
& \multicolumn{1}{c}{时间 (s)} &  文件(KB)			\\
\midrule[1pt]%
\endfirsthead%

\multicolumn{7}{c}{续表~\thetable\hskip1em 实验数据}\\

\toprule[1.5pt]
 测试程序 & \multicolumn{1}{c}{正常运行} & \multicolumn{1}{c}{同步} 
& \multicolumn{1}{c}{检查点}   & \multicolumn{1}{c}{卷回恢复}
& \multicolumn{1}{c}{进程迁移} & \multicolumn{1}{c}{检查点} 	\\
& \multicolumn{1}{c}{时间 (s)} & \multicolumn{1}{c}{时间 (s)}
& \multicolumn{1}{c}{时间 (s)} & \multicolumn{1}{c}{时间 (s)}
& \multicolumn{1}{c}{时间 (s)} &  文件(KB)			\\
\midrule[1pt]%
\endhead%
\hline%

\multicolumn{7}{r}{续下页}%

\endfoot%
\endlastfoot%
CG.A.2 & 23.05   & 0.002 & 0.116 & 0.035 & 0.589 & 32491  \\
CG.A.4 & 15.06   & 0.003 & 0.067 & 0.021 & 0.351 & 18211  \\
CG.A.8 & 13.38   & 0.004 & 0.072 & 0.023 & 0.210 & 9890   \\
CG.B.2 & 867.45  & 0.002 & 0.864 & 0.232 & 3.256 & 228562 \\
CG.B.4 & 501.61  & 0.003 & 0.438 & 0.136 & 2.075 & 123862 \\
CG.B.8 & 384.65  & 0.004 & 0.457 & 0.108 & 1.235 & 63777  \\
MG.A.2 & 112.27  & 0.002 & 0.846 & 0.237 & 3.930 & 236473 \\
MG.A.4 & 59.84   & 0.003 & 0.442 & 0.128 & 2.070 & 123875 \\
MG.A.8 & 31.38   & 0.003 & 0.476 & 0.114 & 1.041 & 60627  \\
MG.B.2 & 526.28  & 0.002 & 0.821 & 0.238 & 4.176 & 236635 \\
MG.B.4 & 280.11  & 0.003 & 0.432 & 0.130 & 1.706 & 123793 \\
MG.B.8 & 148.29  & 0.003 & 0.442 & 0.116 & 0.893 & 60600  \\
LU.A.2 & 2116.54 & 0.002 & 0.110 & 0.030 & 0.532 & 28754  \\
LU.A.4 & 1102.50 & 0.002 & 0.069 & 0.017 & 0.255 & 14915  \\
LU.A.8 & 574.47  & 0.003 & 0.067 & 0.016 & 0.192 & 8655   \\
LU.B.2 & 9712.87 & 0.002 & 0.357 & 0.104 & 1.734 & 101975 \\
LU.B.4 & 4757.80 & 0.003 & 0.190 & 0.056 & 0.808 & 53522  \\
LU.B.8 & 2444.05 & 0.004 & 0.222 & 0.057 & 0.548 & 30134  \\
EP.A.2 & 123.81  & 0.002 & 0.010 & 0.003 & 0.074 & 1834   \\
EP.A.4 & 61.92   & 0.003 & 0.011 & 0.004 & 0.073 & 1743   \\
EP.A.8 & 31.06   & 0.004 & 0.017 & 0.005 & 0.073 & 1661   \\
EP.B.2 & 495.49  & 0.001 & 0.009 & 0.003 & 0.196 & 2011   \\
EP.B.4 & 247.69  & 0.002 & 0.012 & 0.004 & 0.122 & 1663   \\
EP.B.8 & 126.74  & 0.003 & 0.017 & 0.005 & 0.083 & 1656   \\
\bottomrule[1.5pt]
\end{longtable}

贝叶斯公式如~(\ref{equ:chap1:bayes}),其中 $p(y|\mathbf{x})$ 为后验;
$p(\mathbf{x})$ 为先验;分母 $p(\mathbf{x})$ 为归一化因子。
\begin{equation}
\label{equ:chap1:bayes}
p(y|\mathbf{x}) = \frac{p(\mathbf{x},y)}{p(\mathbf{x})}=
\frac{p(\mathbf{x}|y)p(y)}{p(\mathbf{x})} 
\end{equation}

再看一个 \textsf{amsmath} 的例子:
\newcommand{\envert}[1]{\left\lvert#1\right\rvert} 
\begin{equation}\label{detK2}
\det\mathbf{K}(t=1,t_1,\dots,t_n)=\sum_{I\in\mathbf{n}}(-1)^{\envert{I}}
\prod_{i\in I}t_i\prod_{j\in I}(D_j+\lambda_jt_j)\det\mathbf{A}
^{(\lambda)}(\overline{I}|\overline{I})=0.
\end{equation} 

大家在写公式的时候一定要好好看 \textsf{amsmath} 的文档,并参考模板中的用法:
\begin{multline*}%\tag{[b]} % 这个出现在索引中的
\int_a^b\biggl\{\int_a^b[f(x)^2g(y)^2+f(y)^2g(x)^2]
 -2f(x)g(x)f(y)g(y)\,dx\biggr\}\,dy \\
 =\int_a^b\biggl\{g(y)^2\int_a^bf^2+f(y)^2
  \int_a^b g^2-2f(y)g(y)\int_a^b fg\biggr\}\,dy
\end{multline*}

多列公式也是比较常见的情况,比较常用的办法是用align环境实现:

\begin{equation} 
\mathbf{X} = \left(\begin{array}{ccc} 
x_{11} & x_{12} & \ldots \\ 
x_{21} & x_{22} & \ldots \\ 
\vdots & \vdots & \ddots \end{array} \right) 
\end{equation} 

\begin{equation} 
y = \left\{ \begin{array}{ll} 
a & \textrm{if $d>c$}\\ 
b+x & \textrm{in the morning}\\ 
l & \textrm{all day long} 
\end{array} \right. 
\end{equation} 

\begin{equation} 
\left(\begin{array}{c|c} 
1 & 2 \\ 
\hline 3 & 4 \end{array}\right) 
\end{equation}   

\begin{eqnarray}
f(x) & = & \cos x \\ 
f'(x) & = & -\sin x \\ 
\int_{0}^{x} f(y)\,dy & = & \sin x 
\end{eqnarray} 

{\setlength\arraycolsep{2pt} 
\begin{eqnarray} 
\sin x & = & x -\frac{x^{3}}{3!} +\frac{x^{5}}{5!}-{} \nonumber\\ 
	& & {}-\frac{x^{7}}{7!}+{}\cdots 
\end{eqnarray}} 

\begin{theorem}
  \label{chapTSthm:rayleigh solution}
  假定 $X$ 的二阶矩存在:
  \begin{equation}
         O_R(\textbf{x},F)=\sqrt{\frac{\textbf{u}_1^T\textbf{A}\textbf{u}_1} {\textbf{u}_1^T\textbf{B}\textbf{u}_1}}=\sqrt{\lambda_1},
  \end{equation}
  其中 $\textbf{A}$ 等于 $(\textbf{x}-EX)(\textbf{x}-EX)^T$,\textbf{B} 表示协方差阵 $E(X-EX)(X-EX)^T$,$\lambda_1$
$\textbf{u}_1$ 是 $\lambda_1$对应的特征向量。
\end{theorem}

\begin{proof}
 上述优化问题显然是一个 Rayleigh 商问题。我们有
  \begin{align}
     O_R(\textbf{x},F)=\sqrt{\frac{\textbf{u}_1^T\textbf{A}\textbf{u}_1} {\textbf{u}_1^T\textbf{B}\textbf{u}_1}}=\sqrt{\lambda_1},
 \end{align}
 其中 $\lambda_1$ 下列广义特征值问题的最大特征值:
$$
\textbf{A}\textbf{z}=\lambda\textbf{B}\textbf{z}, \textbf{z}\neq 0.
$$
 $\textbf{u}_1$ 是 $\lambda_1$对应的特征向量。结论成立。
\end{proof}
我们构造算法,用于实现非回路故障诊断。

\begin{algorithm}[htbp]
  \caption{非回路故障诊断算法}
  \label{alg53}
  \begin{algorithmic}[1]
    \REQUIRE 信号--部件依赖矩阵$\mathbf{A}$,信号依赖矩阵$\mathbf{S}$,信号状态向量$\alpha$
    \ENSURE 部件状态向量$\gamma$
    \STATE $\mathbf{P}\leftarrow\left(<\alpha>\right)$
    \STATE $\mathbf{S_{a}}\leftarrow\mathbf{P^T}\mathbf{S}\mathbf{P}$
    \FOR{$i=1$ to $S_a$的阶数$m$}
    \STATE $s_i\leftarrow s_i$的第$i$个行向量
    \ENDFOR
    \STATE $\beta_a\leftarrow\lnot \left(s_1\lor s_2\lor \cdots\lor s_m\right)^T$
    \STATE $\beta\leftarrow\mathbf{P}\beta_a$
    \STATE $\gamma\leftarrow\mathbf{A}\beta$
  \end{algorithmic}
\end{algorithm}

有些时候我们需要在论文中引入一段代码,用来衬托正文的内容,或者体现关键思路的实现。
在模板中,统一使用\texttt{listings}宏包,并且设置了基本的内容格式,并建议用户只
使用三个接口,分别控制:编程语言,行号以及边框。简洁达意即可,下面分别举例说明。

首先是设定语言,来一个C的,使用的是默认设置:
\begin{lstlisting}[language=C]
void sort(int arr[], int beg, int end)
{
  if (end > beg + 1)
  {
    int piv = arr[beg], l = beg + 1, r = end;
    while (l < r)
    {
      if (arr[l] <= piv)
        l++;
      else
        swap(&arr[l], &arr[--r]);
    }
    swap(&arr[--l], &arr[beg]);
    sort(arr, beg, l);
    sort(arr, r, end);
  }
}
\end{lstlisting}

当我们需要高亮Java代码,不需要行号,不需要边框时,可以:
\begin{lstlisting}[language=Java,numbers=none,frame=none]
// A program to display the message
// "Hello World!" on standard output

public class HelloWorld {
 
   public static void main(String[] args) {
      System.out.println("Hello World!");
   }
      
}   // end of class HelloWorld
\end{lstlisting}

细心的用户可能发现,行号被放在了正文框之外,事实上这样是比较美观的,如果有些用户希望在正文框架之内布置所有内容,
可以:
\begin{lstlisting}[language=perl,xleftmargin=2em,framexleftmargin=1.5em]
#!/usr/bin/perl
print "Hello, world!\n";
\end{lstlisting}

中文破折号为一个两个字宽垂直居中的直线,输入法直接得到的破折号是两个断开的小短线
(——),这看起来不舒服。所以模板中定义了一个破折号的命令 \verb|\pozhehao|,请看:


为学为师,求实求新\hfill \pozhehao{}首都师范大学校训

