
\begin{cnabstract}
本文介绍了首都师范大学本科生学位论文~\LaTeX~模
板~\texttt{CNUthesisX}~的使用方法;给出了首都师范大学本科生毕业论文(设计)写作规范。

本文研究的内容是红外热波检测中的图像匹配,研究的背景是基于国家 863 计划项
目“红外热波无损检测技术在复合材料研究中的应用”。\par
红外热波无损检测技术是用某种加热办法来激发内部缺陷。大多情况下局部的缺陷
使得热非均匀传播,此时热波将会发生散射和反射等,生成红外热波图像。但当被测物
较大时,不能一次获得整个被测物的红外热波图像。这时就需要多次获得被测物的各个
部分的图像,而对于处在两张图片边缘处的图像就十分不利于检测分析。基于以上需求,
本论文提出了对红外热波图像进行匹配拼接的技术。



\cnkeywords{首都师范大学;毕业论文;\LaTeX{}~模板;红外热波;灰度信息}%%词间用中文状态下“;”分隔
\end{cnabstract}


\begin{abstract}
This paper is an introduction to \LaTeX{} document class
\texttt{CNUthesisX}. A brief guideline for writing thesis is
also included.

This paper aims to present the methodology of image registration in the field of infrared
thermal wave nondestructive inspection. The background of the research is based on National
project 863 “The application of technology of infrared thermal wave nondestructive
inspection in the research of complex material”
\keywords{ Capital Normal University;thesis;\LaTeX{};infrared thermal wave;grey value}%%词间用英文状态下“;”分隔
\end{abstract}
