\chapter{首都师范大学学位论文撰写格式要求}
\label{cha:engorg}
学位论文是表明作者从事科学研究取得的创造性成果和创新见解,并以此为内容撰写的、作为提出申请授予相应的学位评审用的学术论文。现提出学位论文的格式要求,请大家参照执行。

\section{一般要求}
\begin{enumerate}
\item 学位论文中文稿用白色A4纸(210×197mm)标准大小的白纸。每一面的上方和左侧分别留边25mm,下方和右侧应分别留边20mm。

\item 学位论文的页码从“绪论”起(包括绪论、正文、参考文献、附录、致谢等),用阿拉伯数字编码;摘要页、目录、图表目录、符号和缩略词说明等用罗马数字或阿拉伯数字各单独编码。
\end{enumerate}

\section{编排顺序}
\begin{enumerate}
\item 封面
\item “原创性声明”、“授权使用声明”。\\
声明位于论文次页。
\item 中文摘要及中文关键词。\\
选取3—8个中文关键词,另起一行,排在摘要的左下方。
\item 英文摘要及英文关键词。\\
选取3—8个英文关键词,另起一行,排在摘要的左下方。
\item 目录
\item 图和表目录等\\
论文中如图表较多,可以分别列出图表目录置于目录之后。图表的目录应有序号、图表名和页码。
\item 绪论(或引言)
\item 正文
\item 参考文献
\item 附录
\item 致谢
\end{enumerate}
\section{输出要求}

学位论文全部内容应采用计算机编辑,双面打印,使用A4规格纸输出。\par

研究生学位论文完成后,硕士生、博士生均交研究生部。

\section{本学位论文,由研究生部分别提交档案馆、国家图书馆等单位收藏}
\section{论文格式样例}
请见官方Word文档,此处略。
